\documentclass{tufte-handout}

\title{Math 352: Homework 1}
\author{Anthony Brice}

\usepackage{graphicx} % allow embedded images
\setkeys{Gin}{width=\linewidth,totalheight=\textheight,keepaspectratio}
% \graphicspath{{graphics/}} % set of paths to search for images
\usepackage{amsmath, amsthm, amssymb}  % extended mathematics
\usepackage{booktabs} % book-quality tables
\usepackage{units}    % non-stacked fractions and better unit spacing
\usepackage{multicol} % multiple column layout facilities
\usepackage{lipsum}   % filler text
\usepackage{fancyvrb} % extended verbatim environments
  \fvset{fontsize=\normalsize}% default font size for fancy-verbatim environments

% Standardize command font styles and environments
\newcommand{\doccmd}[1]{\texttt{\textbackslash#1}}% command name -- adds backslash automatically
\newcommand{\docopt}[1]{\ensuremath{\langle}\textrm{\textit{#1}}\ensuremath{\rangle}}% optional command argument
\newcommand{\docarg}[1]{\textrm{\textit{#1}}}% (required) command argument
\newcommand{\docenv}[1]{\textsf{#1}}% environment name
\newcommand{\docpkg}[1]{\texttt{#1}}% package name
\newcommand{\doccls}[1]{\texttt{#1}}% document class name
\newcommand{\docclsopt}[1]{\texttt{#1}}% document class option name
\newenvironment{docspec}{\begin{quote}\noindent}{\end{quote}}% command
                                % specification environment

\newcommand{\e}[1]{\ensuremath{\times 10^{#1}}}

\usepackage{hyperref}
\usepackage{natbib}
\renewcommand{\thefootnote}{\fnsymbol{footnote}}

\begin{document}

\maketitle
\section{Section 1.1}
\subsection{Problem 4}
\begin{description}
\item \textit{In racquetball, a player continues to serve as long as
    she is winning; a point is scored only when a player is serving
    and wins the volley. The first player to win $21$ points wins the
    game. Assume that you serve first and have a probability $.6$ of
    winning a volley when you serve and probability $.5$ when your
    opponent serves. Estimate, by simulation, the probability that you
    will win a game.}
\end{description}  

To estimate by simulation our the probability that we will win the
game, we wrote the program, \textbf{Racquetball}\footnote{Source code
  may be found at
  \url{https://github.com/anthonybrice/MATH352/blob/master/hw1/racquetball.c}.}. At
each iteration, the program chooses a random number $m(\omega_i)$ in $(0,1]$. At odd iterations, if $m(\omega_i)$ is less than or equal to
$0.6$, than our score is incremented by $1$. At even iterations, if
$m(\omega_i)$ is greater than $0.5$, then our opponent's score is
incremented by $1$. The game ends when one player reaches $21$ points and
that player is declared the winner. We ran our program $500000$ times,
and found that our probability of winning was about $0.8263$.

\end{document}

