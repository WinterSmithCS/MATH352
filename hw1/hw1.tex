\documentclass{tufte-handout}

\title{Math 352: Homework 1}
\author{Anthony Brice}

\usepackage{graphicx} % allow embedded images
\setkeys{Gin}{width=\linewidth,totalheight=\textheight,keepaspectratio}
% \graphicspath{{graphics/}} % set of paths to search for images
\usepackage{amsmath, amsthm, amssymb}  % extended mathematics
\usepackage{booktabs} % book-quality tables
\usepackage{units}    % non-stacked fractions and better unit spacing
\usepackage{multicol} % multiple column layout facilities
\usepackage{lipsum}   % filler text
\usepackage{fancyvrb} % extended verbatim environments
  \fvset{fontsize=\normalsize}% default font size for fancy-verbatim environments

% Standardize command font styles and environments
\newcommand{\doccmd}[1]{\texttt{\textbackslash#1}}% command name -- adds backslash automatically
\newcommand{\docopt}[1]{\ensuremath{\langle}\textrm{\textit{#1}}\ensuremath{\rangle}}% optional command argument
\newcommand{\docarg}[1]{\textrm{\textit{#1}}}% (required) command argument
\newcommand{\docenv}[1]{\textsf{#1}}% environment name
\newcommand{\docpkg}[1]{\texttt{#1}}% package name
\newcommand{\doccls}[1]{\texttt{#1}}% document class name
\newcommand{\docclsopt}[1]{\texttt{#1}}% document class option name
\newenvironment{docspec}{\begin{quote}\noindent}{\end{quote}}% command
                                % specification environment

\newcommand{\e}[1]{\ensuremath{\times 10^{#1}}} % Macro for scientific notation

% Use fancy symbols for footnotes
\usepackage{hyperref}
\usepackage{natbib}
\renewcommand{\thefootnote}{\fnsymbol{footnote}}
\usepackage{perpage}
\MakePerPage{footnote}

\begin{document}

\maketitle
\section{Section 1.1}
\subsection{Problem 4}
\begin{description}
\item \textit{In racquetball, a player continues to serve as long as
    she is winning; a point is scored only when a player is serving
    and wins the volley. The first player to win $21$ points wins the
    game. Assume that you serve first and have a probability $.6$ of
    winning a volley when you serve and probability $.5$ when your
    opponent serves. Estimate, by simulation, the probability that you
    will win a game.}
\end{description}  

To estimate by simulation the probability that we will win the game,
we wrote the program \textbf{Racquetball}\footnote{
  \url{https://github.com/anthonybrice/MATH352/blob/master/hw1/racquetball.c}}.
At each iteration, the program chooses a random number $m(\omega_i)$
in $(0,1]$. From $m(\omega_i)$, the program calculates if the server
won that volley. If so, the server's score is incremented and she
continues to serve. If not, the other player serves. The game ends
when one player reaches $21$ points and that player is declared the
winner. We ran our program $500000$ times, and found that our
probability of winning was about $0.8263$.

\subsection{Problem 9}
\begin{description}
\item \textit{The \textit{Labouchere system} for roulette is played as
    follows. Write down a list of numbers, usually $1$, $2$, $3$,
    $4$. Bet the sum of the first and last, $1+4 = 5$, on red. If you
    win, delete the first and last numbers from your list. If you
    lose, add the amount that you last bet to the end of your
    list. Then use the new list and bet the sum of the first and last
    numbers (if there is only one number, bet that amount). Continue
    until your list becomes empty. Show that, if this happens, you win
    the sum, $1+2+3+4 = 10$, of your original list. Simulate this
    system and see if you do always stop and, hence, always win. If
    so, why is this not a foolproof gambling system?}
\end{description}

To show that we always win the sum of our list, consider that at any
given moment in the game the sum of the list and our winnings is
always the initial sum of the list. Consider the case when we lose a
round; the debit from our winnings is added to the list. Consider the
case when we win; the credit to our winnings is removed from the
list. Thus we always win the initial sum of the list.

We simulated the system with the program \textbf{Labouchere}\footnote{
  \url{https://github.com/anthonybrice/MATH352/blob/master/hw1/Labouchere.hs}}.
Of course, play always does stop. This is not a foolproof gambling system
because the limit on the sum of the list extends to infinity as play
continues. Thus the player would need a nearly infinite amount of
money to ensure she plays each game until her list is empty.

\subsection{Problem 10}
\begin{description}
\item \textit{Another well-known gambling system is the martingale
    doubling system. Suppose that you are betting on red to turn up in
    roulette. Every time you win, bet $1$ dollar next time. Every time
    you lose, double your previous bet. Suppose that you use this
    system until you have won at least $5$ dollars or you have lost
    more than $100$ dollars. Write a program to simulate this and play
    it a number of times and see how you do. In the book \emph{The
      Newcomes}, W. M. Thackeray remarks ``You have not played as yet?
    Do not do so; above all avoid a martingale if you do.'' Was this
    good advice?}
\end{description}

We wrote a program \textbf{Martingale}\footnote{
  \url{https://github.com/anthonybrice/MATH352/blob/master/hw1/Martingale.hs}}
to simulate the game. Running it for $100000$ iterations found that we
could satisfy the conditions with a $0.94629$ probability. However,
one can clearly see that the conditions are stacked in such a way to
allow this. The betting system favors small gains, so reaching 5
dollars is a fairly trivial task. Conversely winning large amounts is
very difficult since as with any typical gambling game, the odds are
against the player and will become apparent over many repeated
rounds. Thackeray's advice is sound.

\section{Section 1.2}
\subsection{Problem 6}
\begin{description}
\item \textit{A die is loaded in such a way that the probability of
    each face turning up is proportional to the number of dots on that
    face. (For example, a six is three times as likely as a two.) What
    is the probability of getting an even number in one throw?}
\end{description}

To calculate the probability of an even number in one throw, we must
know the probability of each face. Let $m(\omega_1)$ be the probability of
rolling a $1$. Then 
\begin{align*}
  P(\Omega) &= m(\omega_1) + 2m(\omega_1) + 3m(\omega_1) + 4m(\omega_1) +
  5m(\omega_1) + 6m(\omega_1)\\
  1 &= 21m(\omega_1)\\
  m(\omega_1) &= \frac{1}{21}.
\end{align*}
Then the probability of the event of rolling an even face is
\[\frac{2}{21} + \frac{4}{21} + \frac{6}{21} = \frac{12}{21} =
\frac{4}{7}.\]

\subsection{Problem 18}
\begin{description}
\item[(a)] \textit{For events $A_1,\dots, A_n$, prove that
  \[
  P(A_1 \cup \dots \cup A_n) \leq P(A_1) + \dots + P(A_n).
  \]}
\item[(b)] \textit{For events $A$ and $B$, prove that
  \[
  P(A \cap B) \geq P(A) + P(B) - 1.
  \]}
\end{description}

\begin{description}
\item[\emph{(a)}]
  \[
  P(A_1 \cup \dots \cup A_n) \leq P(A_1) + \dots + P(A_n).
  \]

  \begin{proof} This is a corollary of Theorem $3.8$ which states ``to
    find the probability that at least one of $n$ events $A_i$ occurs,
    first add the probability of each event, then subtract the
    probabilities of all two-way intersections, add the probability of
    all three way intersections, and so forth.'' Clearly by adding all
    probabilities, subtracting some intersections, and adding back
    only probabilities of intersections that must be less than those
    already subtracted, the probability of the unions must be less
    than or equal to the sum
    of the probability of the unions.\\
  \end{proof}

\item[\emph{(b)}] 
  \[P(A \cap B) \geq P(A) + P(B) - 1.\]

  \begin{proof}
    We begin with this statement proved in Theorem $1.4$:
    \[P(A \cup B) = P(A) + P(B) - P(A \cap B).\]
    Through substitution and the fact that $P(A \cup B) \leq 1$, we
    have that
    \[1 \geq P(A) + P(B) - P(A \cap B).\]
    We simply rearrange our terms and
    \[P(A \cap B) \geq P(A) + P(B) - 1.\]
  \end{proof}
\end{description}

\subsection{Problem 31}
\begin{description}
\item \textit{A reader of Marilyn vos Savant's column wrote in with
    the following question: \quote{My dad heard this story on the
      radio. At Duke University, two students had received A's in
      chemistry all semester. But on the night before the final exam,
      they were partying in another state and didn't get back to Duke
      until it was over. Their excuse to the professor was that they
      had a flat tire, and they asked if they could take a make-up
      test. The professor agreed, wrote out a test and sent the two to
      separate rooms to take it. The first question (on one side of
      the paper) was worth $5$ points, and they answered it
      easily. Then they flipped the paper over and found the second
      question, worth $95$ points: `Which tire was it?' What was the
      probability that both students would say the same thing? My dad
      and I think it's $1$ in $16$. Is that right?}}
\item[(a)] \textit{Is the answer $1/16$?}
\item[(b)] \textit{The following question was asked of a class of
    students. ``I was driving to school today, and one of my tires
    went flat. Which tire do you think it was?'' The responses were as
    follows: right front, $58\%$, left front, $11\%$, right rear,
    $18\%$, left rear, $13\%$. Suppose that this distribution holds in
    the general population. What is the probability that they will
    give the same answer to the second question?}
\end{description}

\begin{description}
\item[\emph{(a)}] The answer is indeed $1/16$. Let $1$ be the right front
  tire, $2$ be left front, $3$ right rear, and $4$ left rear. We take as sample space
  $\Omega = \{ (i,j) : 1 \leq i,j \leq 4\}$. To determine the size of
  $\Omega$, we note that there are four choices for $i$, and for each
  choice of $i$, there are four choices for $j$, leading to $16$
  different outcomes.
\item[\emph{(b)}] Our sample space $E =
  \{(1,1),(2,2),(3,3),(4,4)\}$. We can easily calculate the
  probability of each outcome in our sample space:
  \begin{align*}
    P(m(1,1)) &= P(1) \cdot P(1) = .58 \cdot .58 = .3364\\
    P(m(2,2)) &= .11 \cdot .11 = .0121\\
    P(m(3,3)) &= .18 \cdot .18 = .0324\\
    P(m(4,4)) &= .13 \cdot .13 = .0169.
  \end{align*}
  Then the sum of all outcomes in $E$ is the probability that the
  students will pick the same tire. Thus
  \[\sum_{\omega \in E} = P(E) = .3978.\]
\end{description}

\end{document}
