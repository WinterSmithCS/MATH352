\documentclass{tufte-handout}

\title{Math 352: Homework 1}
\author{Anthony Brice}

\usepackage{graphicx} % allow embedded images
\setkeys{Gin}{width=\linewidth,totalheight=\textheight,keepaspectratio}
% \graphicspath{{graphics/}} % set of paths to search for images
\usepackage{amsmath, amsthm, amssymb}  % extended mathematics
\usepackage{booktabs} % book-quality tables
\usepackage{units}    % non-stacked fractions and better unit spacing
\usepackage{multicol} % multiple column layout facilities
\usepackage{lipsum}   % filler text
\usepackage{fancyvrb} % extended verbatim environments
  \fvset{fontsize=\normalsize}% default font size for fancy-verbatim environments

% Standardize command font styles and environments
\newcommand{\doccmd}[1]{\texttt{\textbackslash#1}}% command name -- adds backslash automatically
\newcommand{\docopt}[1]{\ensuremath{\langle}\textrm{\textit{#1}}\ensuremath{\rangle}}% optional command argument
\newcommand{\docarg}[1]{\textrm{\textit{#1}}}% (required) command argument
\newcommand{\docenv}[1]{\textsf{#1}}% environment name
\newcommand{\docpkg}[1]{\texttt{#1}}% package name
\newcommand{\doccls}[1]{\texttt{#1}}% document class name
\newcommand{\docclsopt}[1]{\texttt{#1}}% document class option name
\newenvironment{docspec}{\begin{quote}\noindent}{\end{quote}}% command
                                % specification environment

\newcommand{\e}[1]{\ensuremath{\times 10^{#1}}}

\usepackage{hyperref}
\usepackage{natbib}
\renewcommand{\thefootnote}{\fnsymbol{footnote}}

\begin{document}

\maketitle
\section{Section 1.1}
\subsection{Problem 4}
\begin{description}
\item \textit{In racquetball, a player continues to serve as long as
    she is winning; a point is scored only when a player is serving
    and wins the volley. The first player to win $21$ points wins the
    game. Assume that you serve first and have a probability $.6$ of
    winning a volley when you serve and probability $.5$ when your
    opponent serves. Estimate, by simulation, the probability that you
    will win a game.}
\end{description}  

To estimate by simulation the probability that we will win the game,
we wrote the program \textbf{Racquetball}\footnote{Source code may be
  found at
  \url{https://github.com/anthonybrice/MATH352/blob/master/hw1/racquetball.c}.}.
At each iteration, the program chooses a random number $m(\omega_i)$
in $(0,1]$. From $m(\omega_i)$, the program calculates if the server
won that volley. If so, the server's score is incremented and she
continues to serve. If not, the other player serves. The game ends
when one player reaches $21$ points and that player is declared the
winner. We ran our program $500000$ times, and found that our
probability of winning was about $0.8263$.

\subsection{Problem 5}
\begin{description}
\item \textit{The \textit{Labouchere system} for roulette is played as
    follows. Write down a list of numbers, usually $1$, $2$, $3$,
    $4$. Bet the sum of the first and last, $1+4 = 5$, on red. If you
    win, delete the first and last numbers from your list. If you
    lose, add the amount that you last bet to the end of your
    list. Then use the new list and bet the sum of the first and last
    numbers (if there is only one number, bet that amount). Continue
    until your list becomes empty. Show that, if this happens, you win
    the sum, $1+2+3+4 = 10$, of your original list. Simulate this
    system and see if you do always stop and, hence, always win. If
    so, why is this not a foolproof gambling system?}
\end{description}

To show that we always win the sum of our list, consider that at any
given moment in the game the sum of the list and our winnings is
always the initial sum of the list. Consider the case when we lose a
round; the debit from our winnings is added to the list. Consider the
case when we win; the credit to our winnings is removed from the
list. Thus we always win the initial sum of the list.
\end{document}

