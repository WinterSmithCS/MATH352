\documentclass{tufte-handout}

\title{Math 352: Homework 2}
\author{Anthony Brice}

\usepackage{graphicx} % allow embedded images
\setkeys{Gin}{width=\linewidth,totalheight=\textheight,keepaspectratio}
% \graphicspath{{graphics/}} % set of paths to search for images
\usepackage{amsmath, amsthm, amssymb}  % extended mathematics
\usepackage{booktabs} % book-quality tables
\usepackage{units}    % non-stacked fractions and better unit spacing
\usepackage{multicol} % multiple column layout facilities
\usepackage{multirow} % ???
\usepackage{lipsum}   % filler text
\usepackage{fancyvrb} % extended verbatim environments
\fvset{fontsize=\normalsize}% default font size for fancy-verbatim
% environments
\usepackage{enumitem} % for fancy lists

\usepackage{newfloat}
\DeclareFloatingEnvironment[name=Listing]{mylisting}

\newenvironment{listingenv} {
  \begin{mylisting}
} {
  \end{mylisting}
}

\usepackage{subcaption}
\captionsetup{compatibility=false}

\usepackage[scaled]{berasans}
\usepackage[T1]{fontenc}
\usepackage{listings,xcolor}
% \lstloadlanguages{[5.2]Mathematica}
\lstset{language=Mathematica}

\lstset{basicstyle={\sffamily\footnotesize},
  numbers=left,
  numberstyle=\tiny\color{gray},
  numbersep=5pt,
  breaklines=true,
  captionpos={t},
  frame={lines},
  rulecolor=\color{black},
  framerule=0.5pt,
  columns=flexible,
  tabsize=2
}

% Standardize command font styles and environments
\newcommand{\doccmd}[1]{\texttt{\textbackslash#1}}% command name -- adds backslash automatically
\newcommand{\docopt}[1]{\ensuremath{\langle}\textrm{\textit{#1}}\ensuremath{\rangle}}% optional command argument
\newcommand{\docarg}[1]{\textrm{\textit{#1}}}% (required) command argument
\newcommand{\docenv}[1]{\textsf{#1}}% environment name
\newcommand{\docpkg}[1]{\texttt{#1}}% package name
\newcommand{\doccls}[1]{\texttt{#1}}% document class name
\newcommand{\docclsopt}[1]{\texttt{#1}}% document class option name
\newenvironment{docspec}{\begin{quote}\noindent}{\end{quote}}% command
% specification environment

\newcommand{\e}[1]{\ensuremath{\times 10^{#1}}} % Macro for scientific
% notation

% Use fancy symbols for footnotes
\usepackage{hyperref}
\usepackage{natbib}
\renewcommand{\thefootnote}{\fnsymbol{footnote}}
\usepackage{perpage}
\MakePerPage{footnote}

\begin{document}

\maketitle

\section{Exercise 2.1.4}

\textit{Alter the program \textbf{MonteCarlo} to estimate the area
  under the graph of $y = \sin{\pi x}$ inside the unit square by
  choosing $10,000$ points at random. Now calculate the true value of
  this area and use your simulation results to estimate the value of
  $\pi$. How accurate is your estimate?}

\bigskip

My modified program (\lstinline$montecarlo[10000, Sin[Pi#]&, 0, 1, 1]$)
estimated the area to be $.6333$. For the actual area, let $u = \pi
x$, then $du = \pi \; dx$, then
\begin{align*}
  \int_0^1 \sin \pi x \; dx & = \frac{1}{\pi} \int_0^\pi
                              \sin(u) \; du \\
                            &= \left[ - \frac{\cos u}{\pi} \right]_0^\pi \\
                            &= \left[ - \frac{\cos \pi x}{\pi} \right]_0^1 \\
                            &= \frac{2}{\pi}.
\end{align*}

To approximate pi,
\begin{align*}
  \frac{2}{\pi} & \approx .6333 \\
  \frac{\pi}{2} & \approx 1.57903 \\
  \pi & \approx 3.15806.
\end{align*}
My estimate is about $2/100$ off.

\section{Exercise 2.1.5}

\textit{Alter the program \textbf{MonteCarlo} to estimate the area
  under the graph of $y = 1/(x+1)$ in the unit square in the same way
  as Exercise~4. Calculate the true value of this area and use your
  simulation results to estimate the value of $\log{2}$. How accurate
  is your estimate?}

\bigskip

My altered program (\lstinline$montecarlo[10000, 1/(#+1)&, 0, 1, 1]$)
estimated the area to be $.693$. For the true area, let $u = x +
1$, then $du = dx$, then
\begin{align*}
  \int_0^1 \frac{1}{x+1} \; dx &= \int_1^2 \frac{1}{u} \;
                                 du \\
                               &= \left[ \log{u} \right]_1^2 \\
                               &= \log{2} \, .
\end{align*}
Since $\log{2} \approx .69315$, my estimate is pretty good.

\section{Exercise 2.1.9}

\textit{A large number of waiting time problems have an exponential
  distribution of outcomes. We shall see (in Section~5.2) that such
  outcomes are simulated by computing
  $(-1/\lambda)\log{(\mathrm{rnd})}$, where $\lambda > 0$. For waiting
  times produced in this way, the average waiting time is
  $1/\lambda$. For example, the times spent waiting for a car to pass
  on a highway, or the times between emissions of particles from a
  radioactive source, are simulated by a sequence of random numbers,
  each of which is chosen by computing
  $(-1/\lambda)\log{(\mathrm{rnd})}$, where $1/\lambda$ is the average
  time between cars or emissions. Write a program to simulate the
  times between cars when the average time between cars is $30$
  seconds. Have your program compute an area bar graph for these times
  by breaking the time interval from $0$ to $120$ into $24$
  subintervals. On the same pair of axes, plot the function
  $f(x) = (1/30)e^{-(1/30)x}.$ Does the function fit the bar graph
  well?}

\bigskip

\begin{marginfigure}
  \centering
  \includegraphics{barchart}
  \caption{The histogram.}
  \label{fig:barchart}
\end{marginfigure}

\begin{marginfigure}[4cm]
  \includegraphics{exp}
  \caption{The plot of the exponential function.}
  \label{fig:exp}
\end{marginfigure}

We wrote the program \textbf{AvgWait}.  We chose to show two a
histogram and a plot of the exponential function side-by-side because
it appears the \lstinline$Areabargraph$ function provided by the
author is deprecated and fixing it is beyond me. The output of
\lstinline$AvgWait[10000]$ is given in Figures \ref{fig:barchart} and
\ref{fig:exp}.

Since the figures are scaled similarly, one can see that the function
is a very good fit for the graph.

\begin{listingenv}
\caption{The program \mbox{\textbf{AvgWait}}.}
\begin{lstlisting}
AvgWait[n_] :=
 Block[
  {randtime,
   timelist = {}
   },
  For[i = 0, i < n, i++,
   randtime = -30 Log[RandomReal[{0, 1}]];
   timelist = Append[timelist, randtime];
   ];
  Histogram[Select[timelist, 0 <= # <= 120 &], 24]
    Plot[(1/30) Exp[-(1/30) x], {x, 0, 120}]
  ]
\end{lstlisting}
\end{listingenv}


\section{Exercise 2.2.6}

\emph{Assume that a new light bulb will burn out after $t$ hours,
  where $t$ is chosen from $[0,\infty)$ with an exponential density
  \[ f(t) = \lambda e^{-\lambda t}.\]
  In this context, $\lambda$ is often called the \emph{failure rate}
  of the bulb.}
\begin{enumerate}[label=\emph{(\alph*)}]
\item \emph{Assume that $\lambda = 0.01$, and find the probability
    that the bulb will \emph{not} burn out before $T$ hours. This
    probability is often called the \emph{reliability} of the bulb.}
\item \emph{For what $T$ is the reliability of the bulb
    $= 1/2$?}
\end{enumerate}

\bigskip

\begin{enumerate}[label=(\alph*)]
\item To find the probability that the bulb will not burn out
  before $T$ hours, we need to find $F_T(x)$, the cumulative distribution
  function of $f(t)$. By Theorem 2.1,
  \begin{align*}
    F_T(x) = \int_{0}^x \! f(t) \, dt &= \int_{0}^x \!
                                        0.01 e^{-0.01 t} \, dt \\
                                      &= \left[ -e^{-0.01t} \right]_0^x \\
                                      &= 1 - e^{-0.01x} \, .
  \end{align*}
  Then the probability that the bulb will not burn out before $T$
  hours is $1-e^{-0.01T}$.
\item We need to find $t \in T$ such that $F_T(t) = 1/2$. Then
  \begin{align*}
    \frac{1}{2} &= 1 - e^{-0.01t} \\
    \frac{1}{2} &= e^{-0.01t} \\
    \log{\frac{1}{2}} &= -0.01t \\
    t &= -\frac{\log{\frac{1}{2}}}{0.01} \\
                &= 69.3147.
  \end{align*}
\end{enumerate}

\section{Exercise 2.2.12}

\emph{Take a stick of unit length and break it into three pieces,
  choosing the break points at random. (The break points are assumed
  to be chosen simultaneously.) What is the probability that the three
  pieces can be used to form a triangle? \emph{Hint}: The sum of the
  lengths of any two pieces must exceed the length of the third, so
  each piece must have length $< 1/2$. Now use Exercise~8(g).}

\bigskip

\begin{marginfigure}
  \includegraphics{prob}
  \caption{The region our conditions allow.}
  \label{fig:prob}
\end{marginfigure}

Let $X$ and $Y$ be our break points. Our pieces form a triangle if no
piece is longer than any other. Thus, if $0<X<Y<1$, then the lengths
of our line segments are $X$, $Y-X$, and $1-Y$. Then they must satisfy
the conditions
\begin{align*}
  X &< 1 - X \\
  Y &> 1 - Y \\
  Y &< X + \frac{1}{2}.
\end{align*}

Since our sample space is the unit square, we can calculate the area
of the region that our inequalities allow. (See
Figure~\ref{fig:prob}.) Considering the cases in which $X<Y$ and
$Y>X$, we find that the area and thus the probability comes out to
$1/4$.

\section{Exercise 2.2.13*}

\emph{Take a stick of unit length and break it into two pieces,
  choosing the break point at random. Now break the longer of the two
  pieces at a random point. What is the probability that the three
  pieces can be used to form a triangle?}

\bigskip

Consider the two pieces of a broken stick. Let $X$ be the length of
the shorter piece and $Y$ be the length of the longer piece. For the
pieces to form a triangle after the next break, the length of the new
longest piece must be less than or equal to $1/2$. Let $X$ be the
length of the shorter piece. Then the probability of breaking the
other at a length such that the longest piece is less than $1/2$ can
be expressed as%
\[ {X \over 1 - X}. \]

Since $X \in (0, 1/2)$, we want to find the cumulative distribution of
the above probability density. This is expressed%
\[ 2\int_0^{1 \over 2} {X \over 1 - X} \, dX.\]%
Let $u = 1 - X$. Then $1 - u = X$ and $-du = dX$. Then
\begin{align*}
  2\int_{u(0)}^{u \left({1 \over 2} \right) } -{1 - u \over u} \, du
  &= 2\int_1^{1 \over 2} {u - 1 \over u} \, du\\
  &= 2\int_1^{1 \over 2} 1 - {1 \over u} \, du\\
  &= 2{\left[ u - \log u \right]}_1^{1 \over 2}\\
  &= 2\left({1 \over 2} - \log {1 \over 2} - 1\right)\\
  &= 2 \left( -{1 \over 2} + \log 2 \right)\\
  &= -1 + 2 \log 2.
\end{align*}

\section{Exercise 2.2.16*}

\emph{Three points are chosen \emph{at random} on a circle of
  \emph{unit circumference}. What is the probability that the triangle
  defined by these points as vertices has three acute angles?
  \emph{Hint}: One of the angles is obtuse if and only if all three
  points lie on the same semicircle. Take the circumference as the
  interval $[0,1]$. Take one point at $0$ and the others at $B$ and
  $C$.}

\bigskip

Let $A$, $B$, and $C$ our points along the circumference $[0,1]$ of
our circle. Consider the two random variables $A,B$. Regardless of
where they fall along the circumference of unit length, they must
define a unique semicircle. Then without any loss of generality, half
the values in the sample space of $C$ result in the failure of our
desired event and the other half result in its success. Then the
probability that the triangle defined by $A,B,C$ has three acute
angles is $1/2$.

\end{document}

% LocalWords:  vos



%%% Local Variables:
%%% mode: latex
%%% TeX-master: t
%%% End:
