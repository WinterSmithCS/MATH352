\documentclass{tufte-handout}

\title{Math 352: Homework 2}
\author{Anthony Brice}

\usepackage{graphicx} % allow embedded images
\setkeys{Gin}{width=\linewidth,totalheight=\textheight,keepaspectratio}
% \graphicspath{{graphics/}} % set of paths to search for images
\usepackage{amsmath, amsthm, amssymb}  % extended mathematics
\usepackage{booktabs} % book-quality tables
\usepackage{units}    % non-stacked fractions and better unit spacing
\usepackage{multicol} % multiple column layout facilities
\usepackage{lipsum}   % filler text
\usepackage{fancyvrb} % extended verbatim environments
  \fvset{fontsize=\normalsize}% default font size for fancy-verbatim
                              % environments

\usepackage{subcaption}
\captionsetup{compatibility=false}

\usepackage[scaled]{berasans}
\usepackage[T1]{fontenc}
\usepackage{listings,xcolor}
%\lstloadlanguages{[5.2]Mathematica}
\lstset{language=Mathematica}

\lstset{basicstyle={\sffamily\footnotesize},
  numbers=left,
  numberstyle=\tiny\color{gray},
  numbersep=5pt,
  breaklines=true,
  captionpos={t},
  frame={lines},
  rulecolor=\color{black},
  framerule=0.5pt,
  columns=flexible,
  tabsize=2
}

% Standardize command font styles and environments
\newcommand{\doccmd}[1]{\texttt{\textbackslash#1}}% command name -- adds backslash automatically
\newcommand{\docopt}[1]{\ensuremath{\langle}\textrm{\textit{#1}}\ensuremath{\rangle}}% optional command argument
\newcommand{\docarg}[1]{\textrm{\textit{#1}}}% (required) command argument
\newcommand{\docenv}[1]{\textsf{#1}}% environment name
\newcommand{\docpkg}[1]{\texttt{#1}}% package name
\newcommand{\doccls}[1]{\texttt{#1}}% document class name
\newcommand{\docclsopt}[1]{\texttt{#1}}% document class option name
\newenvironment{docspec}{\begin{quote}\noindent}{\end{quote}}% command
                                % specification environment

\newcommand{\e}[1]{\ensuremath{\times 10^{#1}}} % Macro for scientific
                                % notation

% Use fancy symbols for footnotes
\usepackage{hyperref}
\usepackage{natbib}
\renewcommand{\thefootnote}{\fnsymbol{footnote}}
\usepackage{perpage}
\MakePerPage{footnote}

\begin{document}

\maketitle
\section{Section 2.1}
\subsection{Exercise 3}
\begin{description}
\item \textit{Alter the program \textbf{MonteCarlo} to estimate the
    area of the circle of radius $1/2$ with center at $(1/2,1/2)$
    inside the unit square by choosing $1000$ points at random. Compare
    your results with the true value of $\pi/4$. Use your results to
    estimate the value of $\pi$. How accurate is your estimate?}
\end{description}

My modified
program \lstinline$montecarloprime$\footnote{\url{https://github.com/anthonybrice/MATH352/blob/master/hw2/hw2.nb}},
estimated the area to be $.778$. This is about $1/100$ off from
$\pi/4$. Estimating the value $\pi$, \[\pi \approx 4(.778) = 3.112
\,.\]
This puts us about $4/100$ off of the true value of $\pi$.

\begin{lstlisting}
circlep[x_] := {x, 1/2 + Sqrt[x - x^2]}
circlem[x_] := {x, 1/2 - Sqrt[x - x^2]}
montecarloprime = Block[{randpoint,
   count = 0},
  For[i = 0, i <= 1000, i++,
   randpoint = {RandomReal[{0, 1}],
     RandomReal[{0, 1}]};
   If [randpoint[[2]] < circlep[randpoint[[1]]][[2]] && 
     randpoint[[2]] > circlem[randpoint[[1]]][[2]],
    count++
    ];
   ];
  Print[(count/1000) // N];
  ]
\end{lstlisting}

\subsection{Exercise 4}
\begin{description}
\item \textit{Alter the program \textbf{MonteCarlo} to estimate the
    area under the graph of $y = \sin{\pi x}$ inside the unit square by
    choosing $10,000$ points at random. Now calculate the true value
    of this area and use your simulation results to estimate the value
    of $\pi$. How accurate is your estimate?}
\end{description}

My modified program (\lstinline$montecarlo[10000, Sin[Pi#]&, 0, 1, 1]$)
estimated the area to be $.6333$. For the actual area, let $u = \pi
x$, then $du = \pi \; dx$, then
\begin{align*}
  \int_0^1 \sin \pi x \; dx & = \frac{1}{\pi} \int_0^\pi
  \sin(u) \; du \\
  &= \left[ - \frac{\cos u}{\pi} \right]_0^\pi \\
  &= \left[ - \frac{\cos \pi x}{\pi} \right]_0^1 \\
  &= \frac{2}{\pi} \, .
\end{align*}

To approximate pi,
\begin{align*}
  \frac{2}{\pi} & \approx .6333 \\
  \frac{\pi}{2} & \approx 1.57903 \\
  \pi & \approx 3.15806 \, .
\end{align*}
My estimate is about $2/100$ off.

\subsection{Exercise 5}
\begin{description}
\item \textit{Alter the program \textbf{MonteCarlo} to estimate the
    area under the graph of $y = 1/(x+1)$ in the unit square in the
    same way as Exercise~4. Calculate the true value of this area and
    use your simulation results to estimate the value of
    $\log{2}$. How accurate is your estimate?}
\end{description}

My altered program (\lstinline$montecarlo[10000, 1/(#+1)&, 0, 1, 1]$)
 estimated the area to be $.693$. For the true area, let $u = x +
1$, then $du = dx$, then
\begin{align*}
 \int_0^1 \frac{1}{x+1} \; dx &= \int_1^2 \frac{1}{u} \;
 du \\
 &= \left[ \log{u} \right]_1^2 \\
 &= \log{2} \, .
\end{align*}
Since $\log{2} \approx .69315$, my estimate is pretty good.

\subsection{Exercise 9}
\begin{description}
\item \textit{A large number of waiting time problems have an
    exponential distribution of outcomes. We shall see (in
    Section~5.2) that such outcomes are simulated by computing
    $(-1/\lambda)\log{(\mathrm{rnd})}$, where $\lambda > 0$. For
    waiting times produced in this way, the average waiting time is
    $1/\lambda$. For example, the times spent waiting for a car to
    pass on a highway, or the times between emissions of particles
    from a radioactive source, are simulated by a sequence of random
    numbers, each of which is chosen by computing
    $(-1/\lambda)\log{(\mathrm{rnd})}$, where $1/\lambda$ is the
    average time between cars or emissions. Write a program to
    simulate the times between cars when the average time between cars
    is $30$ seconds. Have your program compute an area bar graph for
    these times by breaking the time interval from $0$ to $120$ into
    $24$ subintervals. On the same pair of axes, plot the function
    $f(x) = (1/30)e^{-(1/30)x}.$ Does the function fit the bar graph
    well?}
\end{description}

\begin{marginfigure}[3cm]
  \centering
  \includegraphics{barchart}
  \caption{The histogram}
  \label{fig:barchart}
\end{marginfigure}

\begin{marginfigure}
  \includegraphics{exp}
  \caption{The plot of the exponential function}
  \label{fig:exp}
\end{marginfigure}

We wrote the function \lstinline$AvgWait$,

\begin{lstlisting}
AvgWait[n_] :=
 Block[
  {randtime,
   timelist = {}
   },
  For[i = 0, i < n, i++,
   randtime = -30 Log[RandomReal[{0, 1}]];
   timelist = Append[timelist, randtime];
   ];
  Histogram[Select[timelist, 0 <= # <= 120 &], 24]
    Plot[(1/30) Exp[-(1/30) x], {x, 0, 120}]
  ]
\end{lstlisting}
We chose to show two a histogram and a plot of the exponential
function side-by-side because it appears the \lstinline$Areabargraph$
function provided by the author is deprecated and fixing it is beyond
me. The output of \lstinline$AvgWait[10000]$ is given in Figures
\ref{fig:barchart} and \ref{fig:exp}.

Since the figures are scaled similarly, one can see that the function
is a very good fit for the graph.

\section{Section 2.2}
\subsection{Exercise 6}
\begin{description}
\item \emph{Assume that a new light bulb will burn out after $t$
    hours, where $t$ is chosen from $[0,\infty)$ with an exponential
    density \[ f(t) = \lambda e^{-\lambda t} \, \]
    In this context, $\lambda$ is often called the \emph{failure
      rate} of the bulb.}
  \begin{description}
  \item [(a)] \emph{Assume that $\lambda = 0.01$, and find the probability
    that the bulb will \emph{not} burn out before $T$ hours. This
    probability is often called the \emph{reliability} of the bulb.}
  \item [(b)] \emph{For what $T$ is the reliability of the bulb $= 1/2$?}
  \end{description}
\end{description}

\begin{description}
\item [(a)] To find the probability that the bulb will not burn out
  before $T$ hours, we need to find $F_T(x)$, the cumulative distribution
  function of $f(t)$. By Theorem $2.1$,
  \begin{align*}
    F_T(x) = \int_{0}^x \! f(t) \, dt &= \int_{0}^x \!
    0.01 e^{-0.01 t} \, dt \\
    &= \left[ -e^{-0.01t} \right]_0^x \\
    &= 1 - e^{-0.01x} \, .
  \end{align*}
  Then the probability that the bulb will not burn out before $T$
  hours is $1-e^{-0.01T}$.
\item [(b)] We need to find $t \in T$ such that $F_T(t) = 1/2$. Then
  \begin{align*}
    \frac{1}{2} &= 1 - e^{-0.01t} \\
    \frac{1}{2} &= e^{-0.01t} \\
    \log{\frac{1}{2}} &= -0.01t \\
    t &= -\frac{\log{\frac{1}{2}}}{0.01} \\
    &= 69.3147 \, .
  \end{align*}
\end{description}

\subsection{Exercise 12}
\begin{description}
\item \emph{Take a stick of unit length and break it into three
    pieces, choosing the break points at random. (The break points are
    assumed to be chosen simultaneously.) What is the probability that
    the three pieces can be used to form a triangle? \emph{Hint}: The
    sum of the lengths of any two pieces must exceed the length of the
    third, so each piece must have length $< 1/2$. Now use Exercise $8$(g).}
\end{description}

\begin{marginfigure}
  \includegraphics{prob}
  \caption{The region our conditions allow}
  \label{fig:prob}
\end{marginfigure}

Let $X$ and $Y$ be our break points. Our pieces form a triangle if no
piece is longer than any other. Thus, if $0<X<Y<1$, then the lengths
of our line segments are $X$, $Y-X$, and $1-Y$. Then they must satisfy
the conditions
\begin{align*}
  X &< 1 - X \\
  Y &> 1 - Y \\
  Y &< X + \frac{1}{2} \, .
\end{align*}
Since our sample space is the unit square, we can calculate the area
of the region that our inequalities allow. (See
Figure~\ref{fig:prob}.) Considering the cases in which $X<Y$ and
$Y>X$, we find that the area and thus the probability comes out to
$1/4$.

\end{document}

%  LocalWords:  vos


