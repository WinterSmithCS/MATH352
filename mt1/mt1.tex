\documentclass{tufte-handout}

\title{Math 352: Midterm 1 Take-home Portion}
\author{Anthony Brice \marginnote{I worked with Trevor Scott.}}

\usepackage{graphicx} % allow embedded images
\setkeys{Gin}{width=\linewidth,totalheight=\textheight,keepaspectratio}
% \graphicspath{{graphics/}} % set of paths to search for images
\usepackage{amsmath, amsthm, amssymb}  % extended mathematics
\usepackage{booktabs} % book-quality tables
\usepackage{units}    % non-stacked fractions and better unit spacing
\usepackage{multicol} % multiple column layout facilities
\usepackage{multirow} % ???
\usepackage{lipsum}   % filler text
\usepackage{fancyvrb} % extended verbatim environments
  \fvset{fontsize=\normalsize}% default font size for fancy-verbatim
                              % environments
\usepackage[inline]{enumitem} % for fancy lists

\usepackage[T1]{fontenc}
\usepackage{ccfonts}
\usepackage[euler-digits,euler-hat-accent]{eulervm}
\usepackage[activate={true,nocompatibility},final,tracking=true,kerning=true,spacing=true,factor=1100,stretch=10,shrink=10]{microtype}
% activate={true,nocompatibility} - activate protrusion and expansion
% final - enable microtype; use "draft" to disable
% tracking=true, kerning=true, spacing=true - activate these techniques
% factor=1100 - add 10% to the protrusion amount (default is 1000)
% stretch=10, shrink=10 - reduce stretchability/shrinkability (default
% is 20/20)

\usepackage{newfloat}
\DeclareFloatingEnvironment[name=Listing]{mylisting}

\newenvironment{listingenv} {
  \begin{mylisting}
} {
  \end{mylisting}
}

\usepackage{subcaption}
\captionsetup{compatibility=false}

% \usepackage[scaled]{berasans}
% \usepackage[T1]{fontenc}
% \usepackage{listings,xcolor}
% %\lstloadlanguages{[5.2]Mathematica}
% \lstset{language=Mathematica}

% \lstset{basicstyle={\sffamily\footnotesize},
%   numbers=left,
%   numberstyle=\tiny\color{gray},
%   numbersep=5pt,
%   breaklines=true,
%   captionpos={t},
%   frame={lines},
%   rulecolor=\color{black},
%   framerule=0.5pt,
%   columns=flexible,
%   tabsize=2
% }

% Standardize command font styles and environments
\newcommand{\doccmd}[1]{\texttt{\textbackslash#1}}% command name -- adds backslash automatically
\newcommand{\docopt}[1]{\ensuremath{\langle}\textrm{\textit{#1}}\ensuremath{\rangle}}% optional command argument
\newcommand{\docarg}[1]{\textrm{\textit{#1}}}% (required) command argument
\newcommand{\docenv}[1]{\textsf{#1}}% environment name
\newcommand{\docpkg}[1]{\texttt{#1}}% package name
\newcommand{\doccls}[1]{\texttt{#1}}% document class name
\newcommand{\docclsopt}[1]{\texttt{#1}}% document class option name
\newenvironment{docspec}{\begin{quote}\noindent}{\end{quote}}
% command specification environment

\newcommand{\e}[1]{\ensuremath{\times 10^{#1}}} % Macro for scientific
                                                % notation

\DeclareMathOperator{\Log}{Log}
\DeclareMathOperator{\Arg}{Arg}
\let\Im\relax
\DeclareMathOperator{\Im}{Im}
\let\Re\relax
\DeclareMathOperator{\Re}{Re}

% Use fancy symbols for footnotes
\usepackage{hyperref}
\usepackage{natbib}
\renewcommand{\thefootnote}{\fnsymbol{footnote}}%
\usepackage{perpage}
\MakePerPage{footnote}

\usepackage{manfnt} % ???

\usepackage{mathtools} % for \DeclarePairedDelimiter

\DeclarePairedDelimiter\abs{\lvert}{\rvert}%
\DeclarePairedDelimiter\norm{\lVert}{\rVert}% for nice absolute value
                                            % bars

% For marginnotes in math floats
\newcommand{\filler}[1][10]%
{   \foreach \x in {1,...,#1}
    {   test
    }
}

\def\mathnote#1{%
  \tag*{\rlap{\hspace\marginparsep\smash{\parbox[t]{\marginparwidth}{%
  \footnotesize#1}}}}
}

%\renewcommand{\descriptionlabel}[1]{\hspace{\labelsep}\textsf{#1}}%
\begin{document}

\maketitle

\section{Exercise 5}

\emph{Consider the right isosceles triangle with vertices $(0,0)$,
  $(1,0)$, and $(0,1)$. Let $(X,0)$ be a point chosen uniformly on the
  side of the triangle that lies on the $x$-axis. Let $Y$ be the
  distance from that chosen point $(X,0)$ to vertex $(0,1)$ on the
  $y$-axis.
  \begin{enumerate}[label=(\alph*)]
  \item What is the sample space for $Y$?
  \item What is the probability density function for $Y$?
  \item What is the cumulative distribution function for $Y$?
  \end{enumerate}
}

\bigskip

\begin{enumerate}[label=(\alph*)]
\item Let $\Omega$ be the sample space of $Y$. Note that for all $x
  \in X\setminus\{0,1\}$, the  vertices $(0,0)$, $(X,0)$, and $(0,1)$
  form a scalene triangle. Then $Y$ measures the length of the
  hypotenuse of that scalene triangle. Then
  $\Omega = [1, \sqrt{2}]$.
\item We want to find $f_Y$ such that
  \[
  \int_1^{\sqrt 2} f_Y \, dy = 1.
  \]

  Via (c) (in which we find the cumulative distribution function
  independently of the probability density function) and
  \textbf{Theorem 2.1} from Grinstead, when $1 \leq Y \leq \sqrt 2$
  \begin{align*}
    f_Y(y)
    &= {d \over dy} F_Y(y)\\
    &= {d \over dy} \sqrt{y^2 - 1}\mathnote{Apply the chain rule.}\\
    &= {y \over \sqrt{y^2 - 1}}.
  \end{align*}

  Then
  \[
  f_Y(y) =
  \begin{cases}
    {y \over \sqrt{y^2 - 1}} & \mathrm{for}\ 1 \leq y \leq \sqrt 2\\
    0 & \mathrm{otherwise.}
  \end{cases}
  \]

\item The cumulative distribution function of $f_Y$ is
  \begin{align*}
    F_Y(y)
    &= P(Y \leq y)\mathnote{$Y = \sqrt{1 + X^2}$.}\\
    &= P(\sqrt{1 + X^2} \leq y)\\
    &= P(1 + X^2 \leq y^2)\\
    &= P(X^2 \leq y^2 - 1)\\
    &= P(X \leq \sqrt{y^2 - 1})\mathnote{Note that $X$ is distributed
      uniformly along $[0,1]$. Then $P(X \leq x) = x$.}\\
    &= \sqrt{y^2 - 1}.
  \end{align*}
\end{enumerate}

\section{Exercise 6}

\emph{Let $X$ be uniformly distributed on $(0,1)$.
  \begin{enumerate}[label=(\alph*)]
  \item Find the cumulative distribution function $F_Y$ for the random
    variable $Y = X^{1/\beta}$ where $\beta > 0$.
  \item What is the sample space for $Y$.
  \item Find the probability density function $f$ for $Y$.
  \item For what $\beta$ is $P(Y \leq 1/2) \leq 1/4$?
  \end{enumerate}
}

\bigskip

\begin{enumerate}[label=(\alph*)]
\item
  \begin{align*}
    F_Y(y)
    &= P(Y \leq y)\\
    &= P(X^{1/\beta} \leq y)\\
    &= P(X \leq y^\beta)\\
    &= y^\beta.
  \end{align*}
\item Since $X \in (0,1)$ and $1/\beta > 0$, $Y \in (0,1)$.
\item When $0 < Y < 1$, the probability density function of $Y$ is
  \begin{align*}
    f_Y(y)
    &= {d \over dy} y^\beta\\
    &= \beta y^{\beta - 1}.
  \end{align*}

  Then
  \[
  f_Y(y) =
  \begin{cases}
    \beta y^{\beta - 1} & \mathrm{for}\ 0 < y < 1\\
    0 & \mathrm{otherwise.}
  \end{cases}
  \]
\item Via (a)
  \[
  P\left( Y \leq {1 \over 2} \right) = F_Y\left({1 \over 2} \right) =
  {\left( {1 \over 2} \right)}^\beta.
  \]

  We want to find $\beta$ such that
  \[
  {\left( {1 \over 2} \right)}^\beta \leq {1 \over 4}.
  \]
  Then $\beta \geq 2$ are all $\beta$ such that $P(Y \leq 1/2) \leq 1/4$.
\end{enumerate}

\end{document}

%%% Local Variables:
%%% mode: latex
%%% TeX-master: t
%%% End:
