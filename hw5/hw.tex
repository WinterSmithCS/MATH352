\documentclass{abrice}

\title{Math352: Homework 5}
\author{Anthony Brice}

\begin{document}

\maketitle

\section{Exercise 4.1.5}

\emph{A coin is tossed three times.  Consider the following events\newline
$A$: Heads on the first toss.\newline
$B$: Tails on the second.\newline
$C$: Heads on the third toss.\newline
$D$: All three outcomes the same (HHH or TTT).\newline
$E$: Exactly one head turns up.
\begin{enumerate}[label=(\alph*)]
\item Which of the following pairs of these events are independent?\newline
(1) $A$, $B$\newline
(2) $A$, $D$\newline
(3) $A$, $E$\newline
(4) $D$, $E$
\item Which of the following triples of these events are
independent?\newline
(1) $A$, $B$, $C$\newline
(2) $A$, $B$, $D$\newline
(3) $C$, $D$, $E$
\end{enumerate}}

\bigskip

\begin{enumerate}[label=(\alph*)]
\item Events (1), (2), and (3) are independent.
\item Event (1) is independent.
\end{enumerate}

\section{Exercise 4.1.6}

\emph{From a deck of five cards numbered $2$,~$4$, $6$, $8$, and~$10$,
  respectively, a card is drawn at random and replaced.  This is done
  three times.  What is the probability that the card numbered $2$ was
  drawn exactly two times, given that the sum of the numbers on the
  three draws is~$12$?}

\bigskip

Let $F$ be the event that the card numbered $2$ was drawn exactly two
times. Let $E$ be the event that the sum of the number on the three
draws is $12$. Then we want to calculate $P(F \mid E) = P(F \cap E) /
P(E)$.

Consider $P(F \cap E)$. If the card numbered $2$ is drawn twice and
the sum of the draws is $12$, then the third card must be numbered
$8$. Then $P(F \cap E)$ is the number of permutations with two $2$'s
and one $8$ divided by the total number of permutations of card
sequences, or $P(F \cap E) = 3/ 5^3 = 3/125$.

Consider $P(E)$, the number of permutations which sum to
$12$ % \marginnote{With Haskell's list comprehensions, we calculate that
%   number of permutations quite naturally:\newline%
% \lstinline{length [ (x,y,z) | x <- deck, y <- deck,}\newline%
% \lstinline{\ \ \ \ \ \ \ \ \ \ \ \ \ \ \ \ \ \ \ \ \  z <- deck, x+y+z == 12 ]}}
divided by the total number of permutations. Then $P(E) = 10/125$.

Thus
\begin{align*}
  P(F \mid E)
  &= { {3 \over 125} \over {10 \over 125}}\\
  &= {3 \over 10}.
\end{align*}

\section{Exercise 4.1.13}

\emph{Two cards are drawn from a bridge deck.  What is the probability
  that the second card drawn is red?}

\bigskip

Let $A$ be the event that the second card is red, $B$ the event that
the first card is black, and $C$ the event that the first card is
red. Then $P(A \mid B) = 26/51$ and $P(A \mid C) = 25/51$. Then the
probability of either event is
\[
{1 \over 2} \cdot {26 \over 51} + {1 \over 2} \cdot {25 \over 51} = {1
\over 2}.
\]

\section{Exercise 4.1.17}

\emph{Prove that if $A$ and $B$ are independent so are
\begin{enumerate}[label=(\alph*)]
\item $A$ and $\tilde B$.
\item $\tilde A$ and $\tilde B$.
\end{enumerate}}

\bigskip

\begin{enumerate}[label=(\alph*)]
\item\label{it:1} By the definition of independence
  \begin{align*}
    P(A \cap B)
    &= P(A)P(B)\\
    &= P(A)(1 - P(\tilde{B}))\\
    &= P(A) - P(A)P(\tilde{B}).
  \end{align*}
  Therefore
  \begin{align*}
    P(A)P(\tilde{B})
    &= P(A) - P(A \cap B).
  \end{align*}

  Note that also $P(A \cap \tilde{B}) = P(A) - P(A \cap B)$. Then
  $P(A)P(\tilde{B}) = P(A \cap \tilde{B})$; thus $A$ and $\tilde{B}$
  independent.

\item By~\ref{it:1}, since $A$ and $\tilde{B}$ are independent,
  $\tilde{A}$ and $\tilde{B}$ are independent.
\end{enumerate}

\section{Exercise 4.1.21}

\emph{It is desired to find the probability that in a bridge deal each player
receives an ace.  A student argues as follows.  It does not matter where the
first ace goes.  The second ace must go to one of the other three players and
this occurs with probability $3/4$.  Then the next must go to one of two, an
event of probability $1/2$, and finally the last ace must go to the player who
does not have an ace.  This occurs with probability $1/4$.  The probability that
all these events occur is the product $(3/4)(1/2)(1/4) = 3/32$.  Is this
argument correct?}

\bigskip

This argument is incorrect. The student errs when he or she assumes
that the events are independent and thus the probability of their
occurrence can be calculated as the product of their individual
probabilities.

\section{Exercise 4.1.37*}

\emph{Given that $P(X = a) = r$, $P(\max(X,Y) = a) = s$, and
  $P(\min(X,Y) = a) = t$, show that you can determine $u = P(Y = a)$
  in terms of $r$,~$s$, and~$t$.}

\bigskip

Consider\marginnote{From
  \url{http://www.math.hawaii.edu/~ralph/Classes/371/Answersodd-5-17-03.pdf}.}
\begin{align*}
  P(\max(X,Y) = a) = P(X = a, Y \leq a) + P(X \leq a, Y = a) - P(X =
  a, Y = a).\\
  P(\min(X,Y) = a) = P(X = a, Y < a) + P(X > a, Y = a) + P(X = a, Y = a).
\end{align*}

Then $P(\max(X,Y) = a) + P(\min(X,Y) = a) = P(X = a) + P(Y = a)$. Then
$u = t + s - r$.

\section{Exercise 4.1.49*}

\emph{You are given two urns each containing two biased coins.  The coins in
urn~I come up heads with probability~$p_1$, and the coins in urn~II come up
heads with probability $p_2 \ne p_1$.  You are given a choice of (a)~choosing
an urn at random and tossing the two coins in this urn or (b)~choosing one coin
from each urn and tossing these two coins.  You win a prize if both coins turn
up heads.  Show that you are better off selecting choice (a).}

\bigskip

The probability of winning via choice (a) is
\[
{1 \over 2}p_1^2 + {1 \over 2}p_2^2.
\]

Then
\begin{align*}
  {1 \over 2}p_1^2 + {1 \over 2}p_2^2
  &= {1 \over 2}(p_1^2 + p_2^2)\\
  &> {1 \over 2}(p_1 p_2 + p_1 p_2)\\
  &= p_1 p_2.
\end{align*}
Then $P(\mathrm{(a)}) > P(\mathrm{(b)})$.
\end{document}

%%% Local Variables:
%%% mode: latex
%%% TeX-master: t
%%% End:
