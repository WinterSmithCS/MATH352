\documentclass{tufte-handout}

\title{Math 352: Homework 3}
\author{Anthony Brice}

\usepackage{graphicx} % allow embedded images
\setkeys{Gin}{width=\linewidth,totalheight=\textheight,keepaspectratio}
% \graphicspath{{graphics/}} % set of paths to search for images
\usepackage{amsmath, amsthm, amssymb}  % extended mathematics
\usepackage{booktabs} % book-quality tables
\usepackage{units}    % non-stacked fractions and better unit spacing
\usepackage{multicol} % multiple column layout facilities
\usepackage{multirow} % ???
\usepackage{lipsum}   % filler text
\usepackage{fancyvrb} % extended verbatim environments
  \fvset{fontsize=\normalsize}% default font size for fancy-verbatim
                              % environments
\usepackage{enumitem} % for fancy lists

\usepackage[T1]{fontenc}
%\usepackage[osf]{ebgaramond-maths} % for fancy serif
%\usepackage[cmintegrals,cmbraces]{newtxmath}
%\usepackage[osf,scale=0.9]{librecaslon}
%\usepackage{ebgaramond}
\usepackage{concrete}
\usepackage[euler-digits,euler-hat-accent]{eulervm}
%\usepackage[T1]{fontenc}

%\usepackage{fraktur}

%\usepackage[osf]{AlegreyaSans}
%\renewcommand{\smallcaps}[1]{\sffamily #1}
%\renewcommand*\oldstylenums[1]{{\AlegreyaSansOsF #1}}


% Set up the spacing using fontspec features
% \renewcommand\allcapsspacing[1]{{\addfontfeature{LetterSpace=15}#1}}
% \renewcommand\smallcapsspacing[1]{{\addfontfeature{LetterSpace=10}#1}}

\usepackage{newfloat}
\DeclareFloatingEnvironment[name=Listing]{mylisting}

\newenvironment{listingenv} {
  \begin{mylisting}
} {
  \end{mylisting}
}

\usepackage{subcaption}
\captionsetup{compatibility=false}

% \usepackage[scaled]{berasans}
% \usepackage[T1]{fontenc}
% \usepackage{listings,xcolor}
% %\lstloadlanguages{[5.2]Mathematica}
% \lstset{language=Mathematica}

% \lstset{basicstyle={\sffamily\footnotesize},
%   numbers=left,
%   numberstyle=\tiny\color{gray},
%   numbersep=5pt,
%   breaklines=true,
%   captionpos={t},
%   frame={lines},
%   rulecolor=\color{black},
%   framerule=0.5pt,
%   columns=flexible,
%   tabsize=2
% }

% Standardize command font styles and environments
\newcommand{\doccmd}[1]{\texttt{\textbackslash#1}}% command name -- adds backslash automatically
\newcommand{\docopt}[1]{\ensuremath{\langle}\textrm{\textit{#1}}\ensuremath{\rangle}}% optional command argument
\newcommand{\docarg}[1]{\textrm{\textit{#1}}}% (required) command argument
\newcommand{\docenv}[1]{\textsf{#1}}% environment name
\newcommand{\docpkg}[1]{\texttt{#1}}% package name
\newcommand{\doccls}[1]{\texttt{#1}}% document class name
\newcommand{\docclsopt}[1]{\texttt{#1}}% document class option name
\newenvironment{docspec}{\begin{quote}\noindent}{\end{quote}} % command
% specification environment

\newcommand{\e}[1]{\ensuremath{\times 10^{#1}}} % Macro for scientific
                                % notation

% Use fancy symbols for footnotes
\usepackage{hyperref}
\usepackage{natbib}
\renewcommand{\thefootnote}{\fnsymbol{footnote}}%
\usepackage{perpage}
\MakePerPage{footnote}

\usepackage{manfnt} % ???

%\renewcommand{\descriptionlabel}[1]{\hspace{\labelsep}\textsf{#1}}%

\usepackage[utf8]{inputenc}
\usepackage{microtype}

\begin{document}

\maketitle

\section{Exercise 3.1.6}

\emph{In arranging people around a circular table, we take into
    account their seats relative to each other, not the actual
    position of any one person. Show that $n$ people can be arranged
    around a circular table in $(n - 1)!$ ways.}

\bigskip

Consider that if we arranged $n$ people in a line, we would have $n!$
ways to arrange them. Since the round table has no first or last
position, any single arrangement of $n$ people around a table has $n$
corresponding linear arrangements. Then to compute our solution, we
take $n!$ and divide it by $n$ because $n!$ counts each arrangement
$n$ times. Then
\begin{align*}
\frac{n!}{n} &= \frac{n(n-1)!}{n}\\
&= (n-1)!.
\end{align*}

\section{Exercise 3.1.7}

\emph{Five people get on an elevator that stops at five
  floors. Assuming that each had an equal probability of going to any
  one floor, find the probability that they all get off at different
  floors.}

\bigskip

Start by counting the number of ways in which our success case can
be satisfied. Consider that at the elevator's first stop, we choose
$1$ from $5$ riders to leave, at the second we choose $1$ from $4$, and
so on. Then the number of ways in which our riders can depart and
satisfy our condition for success is $5!$.

Consider that since each person has equal probability of getting off
at any floor, the total number of ways the riders may depart counts a
permutation with repetition, where our set is each floor and we want
to find the number of $5$-tuples. Then that total is $5^5$. Then
the probability that they all get off at different floors is
$5! / 5^5$.

\section{Exercise 3.1.8}

\emph{A finite set $\Omega$ has $n$ elements. Show that if we
    count the empty set and $\Omega$ as subsets, there are $2^n$ subsets
    of $\Omega$.}

\bigskip

Consider the construction of an arbitrary subset of $\Omega$. For each
element in $\Omega$, there are two distinct events: that the element
is included in our subset, or that it is not. Each event is
necessarily independent, so the number of events is $2^n$.

\section{Exercise 3.1.20}

\emph{At a mathematical conference, ten participants are randomly
  seated around a circular table for meals. Using simulation, estimate
  the probability that no two people sit next to each other at both
  lunch and dinner. Can you make an intelligent conjecture for the
  case of $n$ participants when $n$ is large?}

\bigskip

To estimate the probability, we wrote the Haskell program
\textbf{MathConference}%
\footnote{\url{https://github.com/anthonybrice/MATH352/blob/master/hw3/MathConference.hs}},
which makes use of Martin Erwig's functional graph library to generate
random graphs of the seating arrangements for lunch and dinner and
then compares the edges in both graphs for equality. From $50\,000$
trials, we estimated the probability to be about $.08$. By
parameterizing the program on the number of participants $n$, we note
that as $n$ increases so does our probability, but at a vastly reduced
rate.

\section{Exercise 3.1.22*}

\emph{Mr.~Wimply Dimple, one of London's most prestigious watch
  makers, has come to Sherlock Holmes in a panic, having discovered
  that someone has been producing and selling crude counterfeits of
  his best selling watch. The $16$ counterfeits so far discovered bear
  stamped numbers, all of which fall between $1$ and $56$, with the
  largest stamped number equaling $56$, and Dimple is anxious to know
  the extent of the forger's work. All present agree that it seems
  reasonable to assume that the counterfeits thus far produced bear
  consecutive numbers from $1$ to whatever the total number is.}

\emph{``Chin up, Dimple,'' opines Dr.~Watson. ``I shouldn't worry
  overly much if I were you; the Maximum Likelihood Principle, which
  estimates the total number as precisely that which gives the highest
  probability for the series of numbers found, suggests that we guess
  $56$ itself as the total. Thus, your forgers are not a big
  operation, and we shall have them safely behind bars before your
  business suffers significantly.''}

\emph{``Stuff, nonsense, and bother your fancy principles, Watson''
  counters Holmes. ``Anyone can see that, of course, there must be
  quite a few more than $56$ watches---why the odds of our having
  discovered precisely the highest numbered watch made are laughably
  negligible. A much better guess would be twice $56$.''}

\begin{enumerate}[label=\emph{(\alph*)}]
\item \emph{Show that Watson is correct that the Maximum
    Likelihood Principle gives $56$.}
\item \emph{Write a computer program to compare Holmes's and Watson's
    guessing strategies as follows: fix a total of $N$ and choose $16$
    integers randomly between $1$ and $N$. Let $m$ denote the largest
    of these. Then Watson's guess for $N$ is $m$, while Holmes's is
    $2m$. See which is closer to $N$. Repeat this experiment (with $N$
    still fixed) a hundred or more times, and determine the proportion
    of times that each comes closer. Whose seems to be a better
    strategy?}
\end{enumerate}

\bigskip

\begin{enumerate}[label=(\alph*)]
\item Let $n$ be the true number of forged watches. Clearly
  $n \nless 56$. Since we have a uniformly random distribution, $1/n$
  is the likelihood that any watch numbered greater than or equal to
  $56$ is the maximum-numbered watch. Then for all $n$,
  $1/n$ is at its maximum when $n = 56$.
\item We wrote the program
  \textbf{HolmesWatson}%
  \footnote{\url{https://github.com/anthonybrice/MATH352/blob/master/hw3/HolmesWatson.hs}}
  to compare the strategies. With $N = 200$ and $100\,000$ iterations,
  we estimated the probability that Holmes's strategy would come
  closer to be $.00009$. Watson's is a far better strategy whenever
  the sample size is greater than $1$.
\end{enumerate}

\section{Exercise 3.1.23*}

\emph{Barbara Smith is interviewing candidates to be her secretary. As
  she interviews the candidates, she can determine the relative rank
  of the candidates but not the true rank. Thus, if there are six
  candidates and their true rank is $6, 1, 4, 2, 3, 5$, (where $1$ is
  best) then after she had interviewed the first three candidates she
  would rank them $3, 1, 2$. As she interviews each candidate, she
  must either accept or reject the candidate. If she does not accept
  the candidate after the interview, the candidate is lost to her. She
  wants to decide on a strategy for deciding when to stop and accept a
  candidate that will maximize the probability of getting the best
  candidate. Assume that there are $n$ candidates and they arrive in a
  random rank order.}
\begin{enumerate}[label=\emph{(\alph*)}]
\item \emph{What is the probability that Barbara gets the best
    candidate if she interviews all of the candidates? What is it if
    she chooses the first candidate?}
\item \emph{Assume that Barbara decides to interview the first
    half of the candidates and then continue interviewing them until a
    candidate better than any seen so far. Show that she has a better
    than $25$ percent chance of ending up with the best candidate.}
\end{enumerate}

\bigskip

\begin{enumerate}[label= (\alph*)]
\item Since the best candidate has an equal probability of
  appearing in any position in the ordering, the likelihood of either
  event is the same, $1/6$.
\item Consider the case in which the second best candidate
  appears in the first half of the interview and the best candidate
  appears in the second half. The probability of this event is $1/4$
  since the probability of either is $1/2$. Any event of this case
  results in Barbara getting the best candidate. Since we have at
  least $1$ other event in which Barbara hires the best candidate
  (such as $6,5,4,1,2,3$), the probability that this strategy will
  yield the best candidate is greater than $1/4$.
\end{enumerate}

\section{Exercise 3.1.24}

\emph{For the task described in Exercise 3.1.23, it can be shown that
  the best strategy is to pass over the first $k-1$ candidates where
  $k$ is the smallest integer for which}
\[\frac{1}{k} + \frac{1}{k+1} + \cdots + \frac{1}{n-1} \leq 1.\]
\emph{Using this strategy the probability of getting the best
  candidate is approximately $1/e = .368$. Write a program to simulate
  Barbara Smith's interviewing if she uses this optimal strategy,
  using $n=10$, and see if you can verify that the probability of
  success is approximately $1/e$.}

\bigskip

We found that with $n = 10$ our simulation \textbf{InterviewOrder}%
\footnote{\url{https://github.com/anthonybrice/MATH352/blob/master/hw3/InterviewOrder.hs}}
consistently somewhat overestimated the likelihood of success. Over
$200\,000$ iterations, we estimated a likelihood of success to be
$.39774$. We found that as $n$ increased our estimation approached
$1/e$, but we do not have an explanation for the discrepancy.

\end{document}

%%% Local Variables:
%%% mode: latex
%%% TeX-master: t
%%% End:

%  LocalWords:  Watson's
