\documentclass{tufte-handout}

\title{Math 352: Homework 3}
\author{Anthony Brice}

\usepackage{graphicx} % allow embedded images
\setkeys{Gin}{width=\linewidth,totalheight=\textheight,keepaspectratio}
% \graphicspath{{graphics/}} % set of paths to search for images
\usepackage{amsmath, amsthm, amssymb}  % extended mathematics
\usepackage{booktabs} % book-quality tables
\usepackage{units}    % non-stacked fractions and better unit spacing
\usepackage{multicol} % multiple column layout facilities
\usepackage{lipsum}   % filler text
\usepackage{fancyvrb} % extended verbatim environments
  \fvset{fontsize=\normalsize}% default font size for fancy-verbatim
                              % environments


\usepackage{subcaption}
\captionsetup{compatibility=false}

\usepackage[scaled]{berasans}
\usepackage[T1]{fontenc}
\usepackage{listings,xcolor}
%\lstloadlanguages{[5.2]Mathematica}
\lstset{language=Mathematica}

\lstset{basicstyle={\sffamily\footnotesize},
  numbers=left,
  numberstyle=\tiny\color{gray},
  numbersep=5pt,
  breaklines=true,
  captionpos={t},
  frame={lines},
  rulecolor=\color{black},
  framerule=0.5pt,
  columns=flexible,
  tabsize=2
}

% Standardize command font styles and environments
\newcommand{\doccmd}[1]{\texttt{\textbackslash#1}}% command name -- adds backslash automatically
\newcommand{\docopt}[1]{\ensuremath{\langle}\textrm{\textit{#1}}\ensuremath{\rangle}}% optional command argument
\newcommand{\docarg}[1]{\textrm{\textit{#1}}}% (required) command argument
\newcommand{\docenv}[1]{\textsf{#1}}% environment name
\newcommand{\docpkg}[1]{\texttt{#1}}% package name
\newcommand{\doccls}[1]{\texttt{#1}}% document class name
\newcommand{\docclsopt}[1]{\texttt{#1}}% document class option name
\newenvironment{docspec}{\begin{quote}\noindent}{\end{quote}}% command
                                % specification environment

\newcommand{\e}[1]{\ensuremath{\times 10^{#1}}} % Macro for scientific
                                % notation

% Use fancy symbols for footnotes
\usepackage{hyperref}
\usepackage{natbib}
\renewcommand{\thefootnote}{\fnsymbol{footnote}}
\usepackage{perpage}
\MakePerPage{footnote}

\begin{document}

\maketitle
\section{Section 3.1}
\subsection{Exercise 6}
\begin{description}
\item \emph{In arranging people around a circular table, we take into
    account their seats relative to each other, not the actual
    position of any one person. Show that $n$ people can be arranged
    around a circular table in $(n - 1)!$ ways.}
\end{description}

Consider that if we arranged $n$ people in a line, we would have $n!$
ways to arrange them. Since the round table has no first or last
position, any single arrangement of $n$ people around a table has $n$
corresponding linear arrangements. Then to compute our solution, we
take $n!$ and divide it by $n$ to account for the fact that $n!$
counts each arrangement $n$ times. Then
\begin{align*}
\frac{n!}{n} &= \frac{n(n-1)!}{n}\\
&= (n-1)! \, .
\end{align*}

\subsection{Exercise 8}
\begin{description}
\item \emph{A finite set $\Omega$ has $n$ elements. Show that if we
    count the empty set and $\Omega$ as subsets, there are $2^n$ subsets
    of $\Omega$.}
\end{description}

Consider constructing a subset. For each element in $\Omega$, there
are two distinct events: that the element is included or that it is
not. Each event is independent, so the number of events is $2^n$.

\subsection{Exercise 20}
\begin{description}
\item \emph{At a mathematical conference, ten participants are
    randomly seated around a circular table for meals. Using
    simulation, estimate the probability that no two people sit next
    to each other at both lunch and dinner. Can you make an
    intelligent conjecture for the case of $n$ participants when $n$
    is large?}
\end{description}

To estimate the probability, we wrote the program
\textbf{\mbox{MathConference}}. From $50\,000$ trials, it estimated
the probability to be about $.08$. By parameterizing the program on
the number of participants $n$, we note that as $n$ increases so does
our probability, but at a vastly reduced rate.


\end{document}

%%% Local Variables: 
%%% mode: latex
%%% TeX-master: t
%%% End: 
