\documentclass{abrice}

\title{Math 352: Homework 4}
\author{Anthony Brice}



\begin{document}

\maketitle

\section{Exercise 6}

\emph{Charles claims that he can distinguish between beer and ale $75$
  percent of the time.  Ruth bets that he cannot and, in fact, just
  guesses.  To settle this, a bet is made: Charles is to be given ten
  small glasses, each having been filled with beer or ale, chosen by
  tossing a fair coin.  He wins the bet if he gets seven or more
  correct.  Find the probability that Charles wins if he has the
  ability that he claims.  Find the probability that Ruth wins if
  Charles is guessing.}

\bigskip

This is clearly a Bernoulli trials process. Then we want to calculate
$b(10,3/4,7)$ and $b(10,1/2,7)$. Note
$ b(n,p,j) = {n \choose j} p^j q^{n - j}$ where $n$ is the number of
trials, $j$ is the desired number of successes, $p$ is the probability
of success, and $q = 1 - p$.

\begin{align*}
  b\left(10,{3 \over 4},7\right)
  &= {10 \choose 7} {\left( 3 \over 4 \right)}^7 {\left( 1 \over 4
    \right)}^3\\
  &= .250282.
\end{align*}

\begin{align*}
  b \left(10, {1 \over 2}, 7 \right)
  &= {10 \choose 7} {\left( 1 \over 2 \right)}^7 {\left( 1 \over 2
    \right)}^3\\
  &= .117188.
\end{align*}

\section{Exercise 7}

\emph{Show that
  \[
  b(n,p,j) = \frac pq \left(\frac {n - j + 1}j \right) b(n,p,j - 1),
  \]
  for $j \ge 1$. Use this fact to determine the value or values of
  $j$ which give $b(n,p,j)$ its greatest value.  \emph{Hint:} Consider
  the successive ratios as $j$ increases.}

\bigskip

\begin{proof}
  \begin{align*}\mathnote{From \url{http://www.math.hawaii.edu/~ralph/Classes/371/Answersodd-5-17-03.pdf}.}
    {b(n,p,1) \over b(n,p,j-1)}
    &= { {n \choose j} p^j q^{n-j} \over {n \choose j -1} p^{j - 1}
      q^{n - j + 1}}\\
    &= {n! \over j!(n-j)!} \cdot {(n-j + 1)! (j - 1)! \over n!}
      \cdot {p\over q}\\
    &= {p \over q} \cdot {n - j + 1 \over j}.\\
    & \qedhere
  \end{align*}
\end{proof}

Note that
\[
{n - j + 1 \over j} \cdot {p \over q} \geq 1
\]
if and only if $j \leq p(n+1)$. Then $j = \floor{p(n+1)}$ gives the
greatest value when $p(n+1) \notin \mathbb{Z}$. When $p(n+1) \in \mathbb{Z}$,
both $j = p(n+1)$ and $j = p(n+1) - 1$ give the greatest value.

\section{Exercise 10}

\emph{In a ten-question true-false exam, find the probability that a
  student gets a grade of $70$ percent or better by guessing.  Answer
  the same question if the test has $30$ questions, and if the test has
  $50$ questions.}

\bigskip

We want to calculate $b(10,.5,7)$, $b(30,.5,\ceil{.7(30)})$, and
$b(50,.5,\ceil{.7(50)})$.

\begin{align*}
  b(10,.5,7)
  &= {10 \choose 7}.5^7.5^{3}\\
  &= .117188.
\end{align*}

\begin{align*}
  b(30,.5,\ceil{.7(30)})
  &= b(30,.5,21)\\
  &= {30 \choose 21}.5^{21}.5^9\\
  &= .0133246.
\end{align*}

\begin{align*}
  b(50,.5,\ceil{.7(50)})
  &= b(50,.5,35)\\
  &= {50 \choose 35}.5^{35}.5^{15}\\
  &= .00199914.
\end{align*}

\section{Exercise 12}

\emph{A poker hand is a set of 5 cards randomly
chosen from a deck of 52 cards.  Find the probability of a
  \begin{enumerate}[label=(\alph*)]
  \item royal flush (ten, jack, queen, king, ace in a single suit).
  \item straight flush (five in a sequence in a single suit, but not a
    royal flush).
  \item four of a kind (four cards of the same face value).
  \item full house (one pair and one triple, each of the same face
    value).
  \item flush (five cards in a single suit but not a straight or royal
    flush).
  \item straight (five cards in a sequence, not all the same suit).
    (Note that in straights, an ace counts high or low.)
  \end{enumerate}
}

\bigskip

\begin{enumerate}[label=(\alph*)]
\item
\end{enumerate}

\section{Exercise 33*}

\emph{$2n$ balls are chosen at random from a total of $2n$ red balls
  and $2n$ blue balls.  Find a combinatorial expression for the
  probability that the chosen balls are equally divided in color.  Use
  Stirling's formula to estimate this probability.  Using
  \textbf{BinomialProbabilities}, compare the exact value with
  Stirling's approximation for $n = 20$.}

\bigskip

The number of different configurations of $2n$ balls from $4n$ is ${4n
\choose 2n}$. We want to find the number of those that satisfy the
condition of $n$ red balls and $n$ blue balls. Since the total balls
are $1/2$ red and $1/2$ blue, surely half of ${4n \choose 2n}$
satisfies our condition. Then our expression is
\[
{{1 \over 2}{4n \choose 2n} \over {4n \choose 2n}}.
\]

\end{document}

%%% Local Variables:
%%% mode: latex
%%% TeX-master: "hw4.tex"
%%% End:
