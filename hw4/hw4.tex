\documentclass{abrice}

\title{Math 352: Homework 4}
\author{Anthony Brice}


\begin{document}

\maketitle

\section{Exercise 6}

\emph{Charles claims that he can distinguish between beer and ale $75$
  percent of the time.  Ruth bets that he cannot and, in fact, just
  guesses.  To settle this, a bet is made: Charles is to be given ten
  small glasses, each having been filled with beer or ale, chosen by
  tossing a fair coin.  He wins the bet if he gets seven or more
  correct.  Find the probability that Charles wins if he has the
  ability that he claims.  Find the probability that Ruth wins if
  Charles is guessing.}

\bigskip

This is clearly a Bernoulli trials process. Then we want to calculate
$b(10,3/4,7)$ and $b(10,1/2,7)$. Note
$ b(n,p,j) = {n \choose j} p^j q^{n - j}$ where $n$ is the number of
trials, $j$ is the desired number of successes, $p$ is the probability
of success, and $q = 1 - p$.

\begin{align*}
  b\left(10,{3 \over 4},7\right)
  &= {10 \choose 7} {\left( 3 \over 4 \right)}^7 {\left( 1 \over 4
    \right)}^3\\
  &= .250282.
\end{align*}

\begin{align*}
  b \left(10, {1 \over 2}, 7 \right)
  &= {10 \choose 7} {\left( 1 \over 2 \right)}^7 {\left( 1 \over 2
    \right)}^3\\
  &= .117188.
\end{align*}

\section{Exercise 7}

\emph{Show that
  \[
  b(n,p,j) = \frac pq \left(\frac {n - j + 1}j \right) b(n,p,j - 1),
  \]
  for $j \ge 1$. Use this fact to determine the value or values of
  $j$ which give $b(n,p,j)$ its greatest value.  \emph{Hint}: Consider
  the successive ratios as $j$ increases.}

\bigskip

\begin{proof}
  \begin{align*}\mathnote{From \url{http://www.math.hawaii.edu/~ralph/Classes/371/Answersodd-5-17-03.pdf}.}
    {b(n,p,1) \over b(n,p,j-1)}
    &= { {n \choose j} p^j q^{n-j} \over {n \choose j -1} p^{j - 1}
      q^{n - j + 1}}\\
    &= {n! \over j!(n-j)!} \cdot {(n-j + 1)! (j - 1)! \over n!}
      \cdot {p\over q}\\
    &= {p \over q} \cdot {n - j + 1 \over j}.\\
    & \qedhere
  \end{align*}
\end{proof}

Note that
\[
{n - j + 1 \over j} \cdot {p \over q} \geq 1
\]
if and only if $j \leq p(n+1)$. Then $j = \floor{p(n+1)}$ gives the
greatest value when $p(n+1) \notin \mathbb{Z}$. When $p(n+1) \in \mathbb{Z}$,
both $j = p(n+1)$ and $j = p(n+1) - 1$ give the greatest value.

\section{Exercise 10}

\emph{In a ten-question true-false exam, find the probability that a
  student gets a grade of $70$ percent or better by guessing.  Answer
  the same question if the test has $30$ questions, and if the test has
  $50$ questions.}

\bigskip

We want to calculate $b(10,.5,7)$, $b(30,.5,\ceil{.7(30)})$, and
$b(50,.5,\ceil{.7(50)})$.

\begin{align*}
  b(10,.5,7)
  &= {10 \choose 7}.5^7.5^{3}\\
  &= .117188.
\end{align*}

\begin{align*}
  b(30,.5,\ceil{.7(30)})
  &= b(30,.5,21)\\
  &= {30 \choose 21}.5^{21}.5^9\\
  &= .0133246.
\end{align*}

\begin{align*}
  b(50,.5,\ceil{.7(50)})
  &= b(50,.5,35)\\
  &= {50 \choose 35}.5^{35}.5^{15}\\
  &= .00199914
\end{align*}

\section{Exercise 12}

\emph{A poker hand is a set of $5$ cards randomly chosen from a deck of
  52 cards.  Find the probability of a
  \begin{enumerate}[label=(\alph*)]
  \item royal flush (ten, jack, queen, king, ace in a single suit).
  \item straight flush (five in a sequence in a single suit, but not a
    royal flush).
  \item four of a kind (four cards of the same face value).
  \item full house (one pair and one triple, each of the same face
    value).
  \item flush (five cards in a single suit but not a straight or royal
    flush).
  \item straight (five cards in a sequence, not all the same suit).
    (Note that in straights, an ace counts high or low.)
  \end{enumerate}
}

\bigskip

Our general strategy is to divide the number of ways we can deal
the given hand by the number of total hands possible, ${52 \choose 5}$.

\begin{enumerate}[label=(\alph*)]
\item Clearly of all the possible hands, only $4$ give the player a
  royal flush. Then the probability is $4/{52 \choose 5} = .000154\%$.
\item Note that ignoring suit, $9$ distinct hands give the player a
  straight flush. Then the probability is
  ${9 \choose 1}{4 \choose 1}/{52 \choose 5} = .00139\%$.
\item Consider that we choose $1$ rank from $13$ for all suits, and
  then another from the $12$ ranks left for any suit. Then the
  probability is
  \[
  {{13 \choose 1} {4 \choose 4} \cdot {12 \choose 1} {4 \choose 1}
    \over {52 \choose 5}} = .0240\%.
  \]
\item We choose $1$ rank from $13$ for $3$ suits, and another rank
  from $12$ for $2$ suits. Then the probability is
  \[
  {{13 \choose 1} {4 \choose 3} \cdot {12 \choose 1} {4 \choose 2}
    \over {52 \choose 5}} = .1441\%.
  \]
\item We will count any kind of flush and then subtract straight and
  royal flushes. Consider that to deal a flush, we choose $5$ ranks
  from $13$ for $1$ suit. Then the probability is
  \[
  {{13 \choose 5} {4 \choose 1} - {10 \choose 1} {4 \choose 1} \over
    {52 \choose 5}} = .1965\%.
  \]
\item As above, we count any kind of straight, then subtract straight
  and royal flushes. To deal a straight, we choose $1$ from $10$
  hands, and all $5$ cards can be from any suit. Then the probability
  is
  \[
  { {10 \choose 1} {4 \choose 1}^5 - {10 \choose 1} {4 \choose 1}
    \over {52 \choose 5}} = .3925\%.
  \]
\end{enumerate}

\section{Exercise 22}

\emph{How many ways can six indistinguishable letters be put in three
  mail boxes?  \emph{Hint}: One representation of this is given by a
  sequence $|$LL$|$L$|$LLL$|$ where the $|$'s represent the partitions
  for the boxes and the L's the letters.  Any possible way can be so
  described.  Note that we need two bars at the ends and the remaining
  two bars and the six L's can be put in any order.}

\bigskip

Using the formula provided in Exercise~23, $6$ indistinguishable
letters can be put into $3$ mail boxes
\begin{equation*}
  {3 + 6 - 1 \choose 3 - 1 } = 28
\end{equation*}
different ways.

\section{Exercise 33*}

\emph{$2n$ balls are chosen at random from a total of $2n$ red balls
  and $2n$ blue balls.  Find a combinatorial expression for the
  probability that the chosen balls are equally divided in color.  Use
  Stirling's formula to estimate this probability.  Using
  \textbf{BinomialProbabilities}, compare the exact value with
  Stirling's approximation for $n = 20$.}

\bigskip

We choose $n$ balls from $2n$ for red, and the same for blue. Then via
Stirling's formula
\begin{align*}
  { {2n \choose n} \cdot {2n \choose n} \over {4n \choose 2n}}
  &= { {\left( (2n)! \over {(n!)}^2 \right) }^2 \over {(4n)! \over
    {(2n)!}^2}}\\
  &= {\left( (2n)! \over {(n!)}^2 \right) }^2 \cdot { (2n)!^2 \over
    (4n)! }\\
  &= { (2n)! \over {(n!)}^2 } \cdot {(2n)! \over
    {(n!)}^2 } \cdot { (2n)!^2 \over (4n)! }\\
  &= { (2n)!^4 \over {(n!)}^4(4n)! }\\
  &\sim {{ ( {(2n)}^{2n} e^{-2n} \sqrt{ 4 \pi n} ) }^4 \over {(n^n e^{-n}
    \sqrt{2 \pi n})}^4 \cdot ( {(4n)}^{4n} e^{-4n} \sqrt{8 \pi
    n})}\mathnote{By Stirling's formula.}\\
  &= { 2^{4+8n} n^{2+8n} e^{-8n} \pi^2 \over (4 e^{4-n} n^{2+4n} \pi^2)
    \cdot (2^{8 n+3/2} e^{-4 n} n^{4 n+1/2} \sqrt{\pi})}\\
  &= { 2^{4+8n} n^{2+8n} e^{-8n} \pi^2 \over \pi^{5/2} 2^{8 n+7/2}
    e^{-8 n} n^{8 n+5/2}}\\
  &= \sqrt{2 \over \pi n}.
\end{align*}

When $n = 20$
\[
{{40 \choose 20}^2 \over {80 \choose 40}} \sim \sqrt{2 \over 20 \pi} =
{1 \over \sqrt{10 \pi}} \approx .178412.
\]

Via \textbf{BinomialProbabilities[$20,1/2,10,10,\mathrm{True}$]}, the
exact solution is $.1761970520$.

\section{Exercise 35*}

\emph{Prove the following \emph{binomial identity}
  \[
  {2n \choose n} = \sum_{j=0}^n {n \choose j}^2.
  \]
  \emph{Hint}: Consider an urn with $n$ red balls and $n$ blue balls
  inside. Show that each side of the equation equals the number of
  ways to choose $n$ balls from the urn.}

\bigskip

\begin{proof}
  Clearly the left side of the equation counts the number of ways to
  choose $n$ from $2n$. Consider that
  \[
  \sum_{j=0}^n {n \choose j}^2 = \sum_{j=0}^n {n \choose j} {n \choose
  n-j},
  \]
  which counts the same but splits $n$ into two groups. 
\end{proof}
\end{document}

%%% Local Variables:
%%% mode: latex
%%% TeX-master: "hw4.tex"
%%% End:
