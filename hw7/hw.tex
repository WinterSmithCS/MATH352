\documentclass{abrice}

\title{Math 352: Homework 7}
\author{Anthony Brice}

\begin{document}
\maketitle

\section{Exercise 7*}

\emph{Show that, if $X$ and $Y$ are random variables taking on only
  two values each, and if $E(XY) = E(X)E(Y)$, then $X$ and $Y$ are
  independent.}

\bigskip

\begin{proof}
  Choose \marginnote{From %
    \url{http://www.dartmouth.edu/~chance/teaching_aids/books_articles/probability_book/Answersodd.pdf}.}%
  $a,b,c,d$ such that
  \[
  U = {X + a \over b}\ \mathrm{and}\ V = {Y + c \over d}
  \]
  take only values $0$ and $1$. Consider that necessarily $E(UV) =
  E(U)E(V)$. Then
  \[
  E(UV) = P(U=1, V=1) = E(U)E(V) = P(U=1)P(V=1),
  \]
  and
  \begin{align*}
    P(U=1, V=0)
    &= P(U=1) - P(U=1, V=1)\\
    &= P(U=1)(1 - P(V=1))\\
    &= P(U=1)P(V=0).
  \end{align*}

  Similarly,
  \begin{IEEEeqnarray*}{rCl}
    P(U=0, V=1) & = & P(U=0)P(V=1)\\
    P(U=0, V=0) & = & P(U=0)P(V=0).
  \end{IEEEeqnarray*}

  Then $U$ and $V$ are independent, therefore $X$ and $Y$ are independent.
\end{proof}

\section{Exercise 8}

\emph{A royal family has children until it has a boy or until it has
  three children, whichever comes first. Assume that each child is a
  boy with probability $1/2$. Find the expected number of boys in this
  royal family and the expected number of girls.}

\bigskip

Consider that we have only $4$ possible events with easily calculable
probabilities. The event that the family has only $1$ boy has
probability $1/2$, that they have $1$ girl and $1$ boy has probability
$1/4$, and the events of either $2$ girls and $1$ boy or of $3$ girls
and no boy both have probability $1/8$. Then the expected number of
boys is
\[
1 \cdot {1 \over 2} + 1 \cdot {1 \over 4} + 1 \cdot {1 \over 8} + 0
\cdot {1 \over 8} = {7 \over 8},
\]
and the expected number of girls is
\[
0 \cdot {1 \over 2} + 1 \cdot {1 \over 4} + 2 \cdot {1 \over 8} + 3
\cdot {1 \over 8} = {7 \over 8}.
\]

\section{Exercise 13}

\emph{You have $80$ dollars and play the following game. An urn
  contains two white balls and two black balls. You draw the balls out
  one at a time without replacement until all the balls are gone. On
  each draw, you bet half of your present fortune that you will draw a
  white ball until all balls are gone. What is your expected fortune?}

\bigskip

Since we continue playing until all balls are gone, every possible game
leaves us with $45$ dollars. Then our expected fortune is $45$ dollars.

\section{Exercise 15}

\emph{A box contains two gold balls and three silver balls. You are
  allowed to choose successively balls from the box at random. You win
  $1$ dollar each time you draw a gold ball and lose $1$ dollar each
  time you draw a silver ball. After a draw, the ball is not
  replaced. Show that, if you draw until you are ahead by $1$ dollar
  or until there are no more gold balls, this is a favorable game.}

\bigskip

We want to show that with the given conditions the expected value of
our game is greater than $0$. We calculate the value and
probability of all games that satisfy our conditions in Table~\ref{ex15}.

\begin{table}[h]
  \centering
  \label{ex15}
  \begin{tabular}{lcc}
    \toprule
    Sequence & Value (\$) & Probability\\
    \midrule
    $G$ & $1$ & $2/5$\\
    $S,G,G$ & $1$ & $1/10$\\
    $S,S,G,G$ & $0$ & $1/10$\\
    $S,G,S,G$ & $0$ & $1/10$\\
    $S,S,S,G,G$ & $-1$ & $1/10$\\
    $S,G,S,S,G$ & $-1$ & $1/10$\\
    $S,S,G,S,G$ & $-1$ & $1/10$\\
    \bottomrule
  \end{tabular}
  \caption{Let $G$ denote drawing a gold ball and $S$ denote drawing a
  silver ball.}
\end{table}

Then the expected value is $1/5$ dollars, and thus the game is
favorable.

\section{Exercise 17}

\emph{Let $X$ be the first time that a \emph{failure} occurs in an
  infinite sequence of Bernoulli trails with probability $p$ for
  success. Let $p_k = P(X=k)$ for $k = 1,2,\dots$ Show that $p_k =
  p^{k-1} q$ where $q = 1 - p$. Show that $\sum_k p_k = 1$. Show that
  $E(X) = 1/q$. What is the expected number of tosses of a coin
  required to obtain the first tail?}

\bigskip

Consider that $p_k$ equals the probability of $k-1$ successes and $1$
failure, which equals $p^{k-1} q$ where $q = 1 - p$.

Consider
\[
\sum_{k=1}^{\infty} p_k = q \sum_{k=0}^{\infty} p^k = q {1 \over 1 -
  p} = 1.
\]

Consider
\[
E(X) = q \sum_{k=1}^{\infty} k p^{k - 1} = {q \over {(1 - p)}^2} = {1
  \over q}.
\]

Then the expected number of tosses to achieve the first tail is $2$.

\section{Exercise 18}

\emph{Exactly one of six similar keys opens a certain door. If you try
the keys, one after another, what is the expected number of keys you
will have to try before success?}

\bigskip

We have $6$ outcomes all occurring with the same probability. Then the
expected value is
\[
{1 + 2 + 3 + 4 + 5 + 6 \over 6} = 3.5.
\]

\section{Exercise 25*}

\emph{A deck of ESP cards consists of 20 cards each of two types: say
  ten stars, ten circles (normally there are five types).  The deck is
  shuffled and the cards turned up one at a time.  You, the alleged
  percipient, are to name the symbol on each card \emph{before} it is
  turned up.}

\emph{Suppose that you are really just guessing at the cards.  If you
  do not get to see each card after you have made your guess, then it
  is easy to calculate the expected number of correct guesses, namely
  ten.}

\emph{If, on the other hand, you are guessing with information, that
  is, if you see each card after your guess, then, of course, you
  might expect to get a higher score.  This is indeed the case, but
  calculating the correct expectation is no longer easy.}

\emph{But it is easy to do a computer simulation of this guessing with
  information, so we can get a good idea of the expectation by
  simulation.  (This is similar to the way that skilled blackjack
  players make blackjack into a favorable game by observing the cards
  that have already been played.  See Exercise~29.)}

\emph{
  \begin{enumerate}[label=(\alph*)]
  \item First, do a simulation of guessing without information,
    repeating the experiment at least 1000 times.  Estimate the
    expected number of correct answers and compare your result with
    the theoretical expectation.
  \item What is the best strategy for guessing with information?
  \item Do a simulation of guessing with information, using the
    strategy in (b).  Repeat the experiment at least 1000 times, and
    estimate the expectation in this case.
  \item Let $S$ be the number of stars and $C$ the number of circles
    in the deck.  Let $h(S,C)$ be the expected winnings using the
    optimal guessing strategy in (b).  Show that $h(S,C)$ satisfies
    the recursion relation
    \[ h(S,C) = \frac S{S + C} h(S - 1,C) + \frac C{S + C} h(S,C - 1)
    + \frac {\max(S,C)}{S + C}\ ,
    \]
    and $h(0,0) = h(-1,0) = h(0,-1) = 0$.  Using this relation, write
    a program to compute $h(S,C)$ and find $h(10,10)$.  Compare the
    computed value of $h(10,10)$ with the result of your simulation in
    (c).  For more about this exercise and Exercise~26 see Diaconis
    and Graham.
\end{enumerate}}

\bigskip

\begin{enumerate}[label=(\alph*)]
\item My program estimated the value to be $10.004$, which is very
  close to the theoretical
  value. \marginnote{See
    \url{https://github.com/anthonybrice/MATH352/blob/master/hw7/ex25a.hs}.}
\item The best strategy when guessing with information is to always
  guess the type that has appeared less or guess a random type if
  the numbers of appearances are equal.
\item My program estimated the value to $12.275$.\marginnote{See
    \url{https://github.com/anthonybrice/MATH352/blob/master/hw7/ex25c.hs}.}
\item My program calculated the expected value to be
  $12.337731927515208$. \marginnote{See %
    \url{https://github.com/anthonybrice/MATH352/blob/master/hw7/ex25d.hs}.}%
  This is very close to my estimate in (c).
\end{enumerate}

\end{document}

%%% Local Variables:
%%% mode: latex
%%% TeX-master:
%%% End:
